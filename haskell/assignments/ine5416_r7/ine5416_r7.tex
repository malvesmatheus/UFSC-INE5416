\documentclass{article}

\usepackage[utf8]{inputenc}
\usepackage[a4paper, left=20mm, right=20mm, top=20mm, bottom=20mm]{geometry}
\usepackage[colorlinks=true, urlcolor=blue]{hyperref}
\usepackage{amsfonts}
\usepackage{mathtools}

\begin{document}

\subsubsection*{INE5416 - Paradigmas de Programação (2015/2) \\
    Gustavo Zambonin \\
    Relatório 7 - Módulos
}

\textbf{Nota}: todos os excertos de código foram executados com
\texttt{ghci ine5416\_r7.hs} e chamados no interpretador.

\subsubsection*{Questão 1}
Utilizando o conceito de módulos, foi criado um programa que
permite calcular valores para as funções hiperbólicas listadas abaixo,
sendo $e = \sum\limits_{n=0}^\infty \frac{1}{n!} = \frac{1}{0!} +
\frac{1}{1!} + \frac{1}{2!} + \frac{1}{3!} + \dots$ a base do logaritmo
natural, calculada pela soma dos 1000 primeiros termos da série.
\begin{itemize}
    \item $\sinh x = \dfrac{1 - e^{-2x}}{2e^{-x}}$
    \item $\cosh x = \dfrac{1 + e^{-2x}}{2e^{-x}}$
    \item $\tanh x = \dfrac{\sinh x}{\cosh x}$
    \item $\coth x = \dfrac{\cosh x}{\sinh x}$
\end{itemize}
O módulo retorna, comparativamente à funções nativas da linguagem Haskell,
valores significativos apenas até a sexta casa decimal, por conta do
número de computações limitadas da constante $e$, como pode ser
visto abaixo.

\begin{verbatim}
*Hyperbolic> 1/tanh 1
1.3130352854993315
*Hyperbolic> value(Coth 1)
1.3130355
\end{verbatim}
Outras funções podem ser chamadas, respectivamente, por
\texttt{value(Sinh x)}, \texttt{value(Cosh x)} e \texttt{value(Tanh x)},
para $x \in \mathbb{R}$. 

\end{document}
