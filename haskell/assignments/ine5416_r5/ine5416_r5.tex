\documentclass{article}

\usepackage[utf8]{inputenc}
\usepackage[a4paper, left=20mm, right=20mm, top=20mm, bottom=20mm]{geometry}
\usepackage[colorlinks=true, urlcolor=blue]{hyperref}

\begin{document}

\subsubsection*{INE5416 - Paradigmas de Programação (2015/2) \\
    Gustavo Zambonin \\
    Relatório 5 - Análise léxica: Sintaxe (Haskell)
}

\textbf{Nota}: todos os excertos de código foram executados com
\texttt{ghci ine5416\_r5.hs} e chamados no interpretador.

\subsubsection*{Questão 1}
De acordo com
\href{http://rigaux.org/language-study/syntax-across-languages/CntrFlow.html}
{referências} para a sintaxe do Haskell para \textit{switches}, a indentação
correta para o código apresentado seria a seguinte:
\begin{verbatim}
f x = case x of
    0 -> 1
    1 -> 5
    2 -> 2
    _ -> 1

*Main> f 1
5 \end{verbatim}
De modo similar, o código apresentado para a operação \textit{quicksort}
ficaria da seguinte maneira (lista de números aleatórios retirada e adaptada
de \href{https://random.org/integer-sets/?sets=1&num=50&min=1&max=50&commas=on&order=index&format=plain}
{random.org}):
\begin{verbatim}
quicksort :: (Ord a) => [a] -> [a]
quicksort [] = []
quicksort (x:xs)=quicksort lt++ [x]++ quicksort ge where {lt=[y|y<-xs,y<x];ge=[y|y<-xs,y>=x]}

*Main> quicksort [8, 29, 39, 31, 46, 20, 33, 19, 10, 23, 14, 15, 30, 27, 13, 48, 44, 22, 7,
12, 5, 2, 18, 11, 25, 34, 45, 9, 28, 21, 43, 17, 50, 35, 32, 41, 6, 3, 24, 42, 40, 37, 16,
47, 4, 38, 1, 49, 36, 26]
[1,2,3,4,5,6,7,8,9,10,11,12,13,14,15,16,17,18,19,20,21,22,23,24,25,26,27,28,29,30,31,32,33,
34,35,36,37,38,39,40,41,42,43,44,45,46,47,48,49,50] \end{verbatim}

\subsubsection*{Questão 2}
\begin{enumerate}
    \item Uma lista de 1 a 1000 pode ser expressada no interpretador apenas
    digitando \texttt{[1..1000]}, ou armazenada em uma variável a ser chamada
    posteriormente.

    \item Uma progressão aritmética de razão 3 também pode ser obtida de modo
    interativo, digitando \texttt{[1, 4..99]}. Haskell entende que existe uma
    diferença de 3 unidades entre os dois primeiros elementos e reproduz isto até o
    elemento final da lista.

    \item Uma progressão geométrica de razão 2 com 50 termos pode ser generalizada,
    em virtude da sua utilidade para múltiplos primeiros termos, como
    \texttt{pg n = [n*(2**(x-1)) | x <- [1..50]]}. A partir dessa declaração,
    chama-se \texttt{pg 2} se $a_1 = 2$, por exemplo.

    \item A definição estendida da função fatorial não precisa ser declarada
    explicitamente em Haskell. Como a ideia da operação é multiplicar todos os
    elementos de uma lista com $n$ números, começando pelo $1$, então pode-se obter
    este produto com a função \texttt{fat n = product [1..n]}, chamando
    \texttt{fat 5}, por exemplo.
\end{enumerate}

\end{document}
