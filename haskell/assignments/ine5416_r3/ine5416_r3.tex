\documentclass{article}

\usepackage[utf8]{inputenc}
\usepackage[a4paper, left=20mm, right=20mm, top=20mm, bottom=20mm]{geometry}
\usepackage[colorlinks=true, urlcolor=blue]{hyperref}

\begin{document}

\subsubsection*{INE5416 - Paradigmas de Programação (2015/2) \\
    Gustavo Zambonin \\
    Relatório 3 - Estrutura das linguagens
}

\subsubsection*{\texttt{weather.py}}
\begin{itemize}
    \item Este programa foi criado utilizando uma abordagem mais procedimental. Existem diversas aberturas para utilização de funções, sejam estas aninhadas ou não, mas optou-se por uma versão mais simples e legível, que mostra claramente o seu objetivo: um jeito rápido e simples para descobrir a temperatura atual, dada a sua localização arbitrária (a partir de \href{http://www.airlineupdate.com/content_public/codes/airportcodes/airport_icaocodes/airport_icao.htm}{códigos ICAO}). O código-fonte apresenta estruturas como expressões regulares e formatação de \textit{strings}, além de comunicação com a internet, de modo a simplificar o desenvolvimento e também o resultado final, mostrado ao usuário desta maneira:
\end{itemize}

\begin{verbatim}
    [23:35] zambonin@Galileo ~ $ python weather.py SBFL
    The temperature is now 18 C, and the weather is partly cloudy.
\end{verbatim}

\subsubsection*{\texttt{tracking.py}}
\begin{itemize}
    \item De modo contrário ao código anterior, a ideia deste programa é utilizar o máximo possível de abordagens funcionais, como \href{https://docs.python.org/3/howto/functional.html#generator-expressions-and-list-comprehensions}{\textit{list comprehensions}} ("emprestados" de Haskell, são uma forma de retonar uma lista em apenas uma linha, dada certa condição), e suas próprias funções nativas, além de explicitamente dividir o programa em funções. A estratégia geral não difere muito -- procurar informações necessárias no código-fonte de uma página na internet e separar os dados corretamente, novamente com expressões regulares e \textit{slices} em listas. A saída, embora levemente crua, consegue ser perfeitamente inteligível:
\end{itemize}

\begin{verbatim}
    [23:54] zambonin@Galileo ~ $ python tracking.py PI906191285BR
    [['08/09/2015 14:09',
      'AC MARAVILHA - Maravilha/SC',
      'Postado depois do horário limite da agência',
      'Objeto sujeito a encaminhamento no próximo dia útil  '],
     ['08/09/2015 14:17',
      'AC MARAVILHA - Maravilha/SC',
      'Encaminhado',
      'Encaminhado para ENTREPOSTO CHAPECO - Chapeco/SC'],
     ['08/09/2015 18:02',
      'ENTREPOSTO CHAPECO - Chapeco/SC',
      'Encaminhado',
      'Em trânsito para CTE FLORIANOPOLIS/GTURN3 (BLUMENAU) - Blumenau/SC'],
     ['09/09/2015 16:56',
      'CTE FLORIANOPOLIS/GTURN3 (BLUMENAU) - Blumenau/SC',
      'Encaminhado',
      'Em trânsito para CTE BRASILIA - Brasilia/DF']]
\end{verbatim}

\end{document}
