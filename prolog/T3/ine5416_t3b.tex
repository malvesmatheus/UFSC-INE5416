\documentclass{article}

\usepackage[utf8]{inputenc}
\usepackage[a4paper, left=20mm, right=20mm, top=20mm, bottom=20mm]{geometry}
\usepackage{amsfonts}
\usepackage{mathtools}

\begin{document}

\subsubsection*{INE5416 - Paradigmas de Programação (2015/2) \\
    Gustavo Zambonin, Lucas Kramer de Sousa,
    Marcello da Silva Klingelfus Junior \\
    Programação Lógica - T3B
}

\textbf{Nota}: todos os excertos de código foram executados com
\texttt{swipl -qs ine5416\_t3b.pl} e chamados no interpretador. Recomenda-se,
também, utilizar aspas duplas em nomes de arquivos como argumentos.

\begin{itemize}

\item \texttt{load}

Este predicado é responsável por carregar o conteúdo do arquivo
\texttt{database.pl}, necessário para o discernimento de imagens pelo programa
a partir de seus momentos de Hu. É inicializado na compilação/interpretação do
código para praticidade do usuário.

\item \texttt{write\_database}

Escreve modificações (adição de novas imagens, por exemplo) no banco de dados.

\item \texttt{new}

Habilita a adição de novas imagens de duas maneiras: a partir de um arquivo PGM
(o identificador deste será derivado do nome de arquivo automaticamente), ou
com a inserção dos momentos de Hu e identificador manualmente (geralmente
utilizado internamente pelo programa).

\item \texttt{check\_database}

Comanda a lógica de verificação de proximidade de momentos de Hu. Após a leitura
do arquivo e cálculo do somatório de momentos de Hu, o predicado busca o menor
destes já existente no banco de dados e pergunta ao usuário se a descrição da
imagem é convergente com o resultado de comparação. Se a resposta for positiva,
uma nova linha será adicionada ao banco de dados, com o mesmo identificador
apresentado. Do contrário, é responsabilidade do usuário descrever o novo tipo
de imagem, que eventualmente também será adicionada ao banco de dados. (Note a
adição de pontos finais às entradas do usuário.)
\begin{verbatim}
?- check_database("ufsc_neg.pgm").
(y/n) Is this your image? Id: cameraman
|: n.
Identify it:
|: ufsc_neg.
Image added.
true.

?- check_database("ufsc_neg.pgm").
(y/n) Is this your image? Id: ufsc_neg
|: y.
No operation is needed.
true.
\end{verbatim}

\item \texttt{min\_pos}

Predicado auxiliar que retorna o índice da imagem no banco de dados, de acordo
com o menor somatório de momentos de Hu retornado pela comparação.

\item \texttt{compare}

Responsável por retornar uma lista contendo o somatório de cada conjunto de
momentos de Hu de uma imagem no banco de dados, utilizado como critério para
definir se duas imagens são parecidas o bastante.

\item \texttt{distance}

Implementa a distância Euclidiana entre dois pontos em um espaço, utilizada
como medidor de divergência entre imagens no banco de dados. A distância é dada
pela fórmula abaixo.
\begin{gather*}
d(p,q) = d(q,p)
= \sqrt{(q_1 - p_1)^2 + (q_2 - p_2)^2 + \dots + (q_n - p_n)^2}
= \sqrt{\sum\limits_{i=1}^{n} (q_i - p_i)^2}
\end{gather*}

\end{itemize}

\end{document}
